\chapter{Background} \label{chap:background}
This chapter is focused on making the reader familiar with concepts used throughout this report.
First, an introduction to computational complexity is given to establish a mathematical framework to describe the efficiency of computer algorithms.
Second, the basic ideas of quantum information theory are presented.
Finally, an overview of quantum computation is given.

\section{Computational Complexity}
In computer science, there seems to be a fundamental limit to what problems we can solve.
Some problems seem to be inherently uncomputable: there exists no general solution that does not go into an infinite loop for certain inputs~\cite{church1936note, turing1937computable}.
This report will not go further into what problems are computable and uncomputable.
Rather, it will look at the computational efficiency of certain algorithms: how much resources are required to solve a problem?

The time and space taken by an algorithm generally grows as the size of the input grows.
Because of this, it is traditional to describing the efficiency of an algorithm as a function of the size of its input.
The notion for input size here depends on the context of the problem.
For example, when computing the discrete Fourier transform, the input size refers to the dimension of the input vector.
When talking about a problem like integer multiplication however, it is more fitting to talk about the input size as the amount of bits needed to represent the input in binary.


% the set of computable problems does not depend on the computational models - all known computational models can simulat eachother (though not necessarily efficiently)


\section{Quantum Information}

\section{Quantum Computation}