\chapter{QAOA for Max-Cut} \label{chap:qaoa-maxcut}
The aim of the \gls{maxcut} problem is to find a set of vertices that maximizes the sum of the weights of the edges that are cut.
It is a widely studied problem known to be $\NP$-hard.
Formally, given an undirected graph $G = (V, E)$ with $n = |V|$ vertices and non-negative weights $w_{j, k} = w_{k, j}$ on the edges $(j, k) \in E$, one looks to bipartition $V$ into two sets $S \subseteq V$ and $\bar{S} = V \setminus S \subseteq V$ such that the objective function $L$ is maximized:
\begin{equation} \label{eqn:max-cut-objective}
L(z) = \sum_{(j, k) \in E} w_{j, k}\left[z_j\left(1 - z_k\right) + z_k(1 - z_j)\right].
\end{equation}
Here $z \in \{0, 1\}^n$ is a bit string that describes the bipartition as follows: if a vertex $j$ is in partition $S$, then $z_j = 0$, and if a vertex $j$ is in partition $\bar{S}$ then $z_j = 1$.
For example, \Cref{fig:maxcut-5-example} shows an example of a maximum cut on a graph with five vertices.
The partition of the visualized cut can be described as the bit string $z = 00101$.
\begin{figure}[ht]
    \centering
    \includegraphics[width=0.4\linewidth]{figures/maxcut_5_graph_cut_example.pdf}
    \caption[\Gls{maxcut} example on a graph with five vertices and unit weights.]{
        \Gls{maxcut} example on a graph with five vertices and unit weights.
        The vertices are partitioned into two sets visualized as black and gray.
        The cut shown is a maximum cut with $L(z) = 5$, which can be thought of as the number of edges cut (shown in red).
    }
    \label{fig:maxcut-5-example}
\end{figure}

The \gls{qaoa} algorithm can be used for solving \gls{maxcut} problems by assigning a vertex $j \in V$ to a qubit $\ket{q_j}$.
A qubit $\ket{q_j}$ is in state \ket{0} if a vertex $j$ is in partition $S$, and state \ket{1} if vertex $j$ is in partition $\bar{S}$.
To encode the objective function from \Cref{eqn:max-cut-objective}, note that the objective function can be rewritten as follows:
\begin{equation}
L(z) = \frac{1}{2} \sum_{(j, k) \in E} w_{j, k}(1 - z_jz_k),
\end{equation}
where $z_j \in \{-1, 1\}$ for $j \in V$.
This objective function can be represented by the following problem Hamiltonian:
\begin{equation} \label{eqn:problem-hamiltonian}
H_L = \frac{1}{2} \sum_{(j, k) \in E} w_{j, k}\left(I - Z^{(j)}Z^{(k)}\right).
\end{equation}
This gives the problem unitary
\begin{equation} \label{eqn:problem-unitary}
U_L(\gamma) = e^{-i\gamma H_L} = \prod_{(j, k) \in E} e^{-i\gamma w_{j, k}(I - Z^{(j)}Z^{(k)})/2},
\end{equation}
and the standard mixer unitary as defined in \Cref{sec:qaoa}:
\begin{equation} \label{eqn:mixer-unitary}
U_B(\beta) = e^{-i\beta H_B} = \prod_{j \in V} e^{-i\beta X^{(j)}}.
\end{equation}

\section{Implementation} \label{sec:qaoa-implementation}
The \gls{qaoa} was implemented in Python using the Project Q~\cite{steiger2018projectq}, Quantum Inspire~\cite{quantuminspire}, and SciPy~\cite{scipy} frameworks to solve the \gls{maxcut} problem on the graph from \Cref{fig:maxcut-4-graph}.
The optimal solutions for this graph are $z = 0101$ and $z = 1010$ with $L(z) = 4$.
\begin{figure}[ht]
    \centering
    \includegraphics[width=0.375\linewidth]{figures/maxcut_4_graph.pdf}
    \caption{
        Undirected 2-regular graph $G = (V, E)$ with $n = 4$ vertices $V = \{0, 1, 2, 3\}$ and 4 edges $E = \{(0,1), (0,3), (1,2), (2,3)\}$ with unit weight $w_{j, k} = w_{k, j} = 1$.
    }
    \label{fig:maxcut-4-graph}
\end{figure}
A $ZZ$ interaction $e^{-i\gamma w_{j, k}(I - Z^{(j)}Z^{(k)})/2}$ from the problem unitary $U_L(\gamma)$ (\Cref{eqn:problem-unitary}) is implemented as follows:
\begin{figure}[H]
    \[
    \Qcircuit @C=1em @R=1em @!R {
        & \lstick{\ket{q_j}} & \ctrl{1} & \qw & \ctrl{1} & \qw \\
        & \lstick{\ket{q_k}} & \targ & \gate{R_z(-\gamma w_{j, k})} & \targ & \qw
    }
    \]
\end{figure}
\noindent
The $X$ interaction $e^{-i\beta X^{(j)}}$ from the mixer unitary $U_B(\beta)$ (\Cref{eqn:mixer-unitary}) is implemented as a $R_x$ gate:
\begin{figure}[H]
    \[
    \Qcircuit @C=1em @R=1em @!R {
        & \lstick{\ket{q_j}} & \gate{R_x(2\beta)} & \qw \\
    }
    \]
\end{figure}
\noindent
The complete quantum circuit for the \gls{qaoa} for \gls{maxcut} on this graph is shown in \Cref{fig:qaoa-circuit} and the respective cQASM for $p = 1$ is shown in \Cref{fig:qaoa-cqasm}.

\begin{figure}[H]
    \begin{adjustwidth}{-1cm}{-1cm}
    \[
    \Qcircuit @C=0.275em @R=0.6em @!R {
        & & & & & \lstick{\ket{q_0}} & \gate{H} & \qw & \ctrl{1} & \qw & \ctrl{1} & \qw & \ctrl{3} & \qw & \ctrl{3} & \qw & \qw & \qw & \qw & \qw & \qw & \qw & \qw & \qw & \gate{R_x(2\beta)} & \qw & \qw & \meter & \cw \\
        & & & & & \lstick{\ket{q_1}} & \gate{H} & \qw & \targ & \gate{R_z(-\gamma w_{0, 1})} & \targ & \qw & \qw & \qw & \qw & \qw & \ctrl{1} & \qw & \ctrl{1} & \qw & \qw & \qw & \qw & \qw & \gate{R_x(2\beta)} & \qw & \qw & \meter & \cw \\
        & & & & & \lstick{\ket{q_2}} & \gate{H} & \qw & \qw & \qw & \qw & \qw & \qw & \qw & \qw & \qw & \targ & \gate{R_z(-\gamma w_{1, 2})} & \targ & \qw & \ctrl{1} & \qw & \ctrl{1} & \qw & \gate{R_x(2\beta)} & \qw & \qw & \meter & \cw \\
        & & & & & \lstick{\ket{q_3}} & \gate{H} & \qw & \qw & \qw & \qw & \qw & \targ & \gate{R_z(-\gamma w_{0, 3})} & \targ & \qw & \qw & \qw & \qw & \qw & \targ & \gate{R_z(-\gamma w_{2, 3})} & \targ & \qw & \gate{R_x(2\beta)} \gategroup{1}{9}{4}{25}{1em}{--} & \qw & \qw & \meter & \cw \\
        & & & & & & & & & & & & & \hspace{5cm} p \mbox{ times}
    }
    \]
    \end{adjustwidth}
    \caption[Quantum circuit for the $p$-layer \gls{qaoa} on the graph from \Cref{fig:maxcut-4-graph}.]{
        Quantum circuit for the $p$-layer \gls{qaoa} on the graph from \Cref{fig:maxcut-4-graph}.
        Each vertex $j \in V$ is represented by qubit $\ket{q_j}$.
        The circuit starts by preparing an equal superposition state, after which the problem unitary $U_L(\gamma)$ and mixer unitary $U_B(\beta)$ are applied $p$ times.
        In general, the depth of the circuit is $p(3m + n)$, where $p$ is the number of layers, $m$ is the number of edges, and $n$ is the number of vertices. 
    }
    \label{fig:qaoa-circuit}
\end{figure}

\begin{figure}[H]
    \begin{minted}[frame=lines,framesep=2mm,linenos,baselinestretch=0.95]{vhdl}
version 1.0

qubits 5

H q[0]
H q[1]
H q[2]
H q[3]
SWAP q[2], q[1]
CNOT q[3], q[2]
Rz q[2], 3.902484
CNOT q[3], q[2]
CNOT q[2], q[1]
Rz q[1], 3.902484
SWAP q[2], q[3]
CNOT q[2], q[0]
Rz q[0], 3.902484
SWAP q[2], q[1]
CNOT q[3], q[2]
SWAP q[0], q[2]
CNOT q[1], q[2]
CNOT q[0], q[2]
Rz q[2], 3.902484
CNOT q[0], q[2]
Rx q[0], 5.156352
Rx q[1], 5.156352
Rx q[2], 5.156352
Rx q[3], 5.156352
    \end{minted}
    \caption[\acrshort{cqasm} for the $p$-layer \gls{qaoa} on the graph from \Cref{fig:maxcut-4-graph}.]{
        \acrshort{cqasm} for the $1$-layer \gls{qaoa} on the graph from \Cref{fig:maxcut-4-graph} with $\gamma = 3.902484$ and $\beta = 5.156352$.
        Swap gates are added to deal with the limited qubit connectivity of the Starmon-5 hardware back-end.
    }
    \label{fig:qaoa-cqasm}
\end{figure}


The quantum circuit is implemented using the Qiskit quantum computing library.
To find the optimal parameters $\vec{\gamma}_\text{opt}, \vec{\beta}_\text{opt}$ a classical optimizer provided by SciPy is used.
The relevant cost function is
\begin{equation}
C_p(\vec{\gamma}, \vec{\beta}) = \bra{\vec{\gamma}, \vec{\beta}}H_L\ket{\vec{\gamma}, \vec{\beta}},
\end{equation}
and we look to solve the optimization problem
\begin{equation}
\vec{\gamma}_\text{opt}, \vec{\beta}_\text{opt} = \argmax_{\vec{\gamma}, \vec{\beta}} C_p(\vec{\gamma}, \vec{\beta}).
\end{equation}
We then prepare the state $\ket{\vec{\gamma}_\text{opt}, \vec{\beta}_\text{opt}}$ and measure multiple times in the computational basis to extract the solution, which is the bit string with the highest probability.
If there are $n$ optimal solutions, they can be extracted by choosing the $n$ bit strings with the highest probabilities. 

\section{Results}
The implementation from the previous section is run on two different Quantum Inspire back-ends: the QX simulator back-end and the Starmon-5 \gls{qpu} back-end.
The cost function is optimized for 30 iterations using the \gls{cobyla} algorithm.
For the quantum circuit executions during optimization 1024 shots were used, and for the final measurement 4096 shots were used.
The results from the experiments are shown in \Cref{fig:qaoa-results}.
The QX simulator back-end manages to find a local minimum with a final approximation ratio of $0.76$ and high probabilities of measuring the optimal solutions $z = 1010$ and $z = 0101$.
The Starmon-5 \gls{qpu} back-end reaches a final approximation ratio of $0.63$ and manages to reach a final state which measures an optimal solution $z = 0101$ with high probability.
The performance difference between the simulator and \gls{qpu} back-end was expected: while \glspl{hqca} have some robustness against noise, their performance on real hardware will still be significantly worse from idealized noise-free simulations.
Further performance improvement could be achieved on \glspl{qpu} by choosing a classical optimizer that is more robust against noise as discussed in \cite{lavrijsen2020classical, sung2020exploration}, but such work is beyond the scope of this report.

\begin{figure}[ht]
    \centering
    \begin{subfigure}{.49\textwidth}
        \centering
        \includegraphics[width=1\linewidth]{figures/qaoa_maxcut_n4_p1_qx_optimization.pdf}
    \end{subfigure}
    \hfill
    \begin{subfigure}{.49\textwidth}
        \centering
        \includegraphics[width=1\linewidth]{figures/qaoa_maxcut_n4_p1_qx_probs.pdf}
    \end{subfigure}

    \begin{subfigure}{.49\textwidth}
        \centering
        \includegraphics[width=1\linewidth]{figures/qaoa_maxcut_n4_p1_starmon_optimization.pdf}
    \end{subfigure}
    \hfill
    \begin{subfigure}{.49\textwidth}
        \centering
        \includegraphics[width=1\linewidth]{figures/qaoa_maxcut_n4_p1_starmon_probs.pdf}
    \end{subfigure}

    \caption[Experimental results of the running the \gls{qaoa} to solve \gls{maxcut} problem on the graph from \Cref{fig:maxcut-4-graph}.]{
        Experimental results of the running \gls{qaoa} to solve the \gls{maxcut} problem on the graph from \Cref{fig:maxcut-4-graph}.
        The left column plots the approximation ratio $C_p(\vec{\gamma}, \vec{\beta})/L_\text{max}$ over 30 iterations, and the right column plots the final probabilities of measuring the possible solution bit strings after optimization.
        The top row contains results from the QX simulator back-end, and the bottom row contains the results from the Starmon-5 \gls{qpu} back-end.
    }
    \label{fig:qaoa-results}
\end{figure}